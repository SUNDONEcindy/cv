% !TEX encoding = UTF-8 Unicode
%!TEX TS-program = xelatex

% Template from https://www.sharelatex.com/templates/cv-or-resume/fancy-cv

\documentclass[]{friggeri-cv}
\addbibresource{bibliography.bib}

\begin{document}
\header{Damien} {DUPORTAL}
       {Ingénieur informaticien, Développeur, DevOps}


% In the aside, each new line forces a line break
\begin{aside}
  \section{Personnel}
    25 rue Desaix
    69003 Lyon
    France
    ~
    Permis B
    ~
    +33(0)6 50 83 37 76
    \href{mailto:damien.duportal@gmail.com}{damien.duportal@gmail.com}
    \href{http://github.com/dduportal}{http://github.com/dduportal}
  \section{Langues} 
    Espagnol : courant
    Anglais : bon
\end{aside}

\section{Expérience}

\begin{entrylist}
  \entry
    {03/2014 à Ajd.}
    {EPSI Lyon}
    {Enseignement ponctuel}
    {\emph{Module de cours + TP de 20h autour de la haute disponibilité. Utilisation de Docker à des fins de simulations d'architectures.}}
  \entry
    {12/2012 à Ajd.}
    {Worldline}
    {Innovation Labs - Unité PST}
    {\emph{Laboratoire d'innovation (1j / semaine) en dehors du contexte opérationnel. Etude et production d'outils type DevOps, présentation et formation à ce coutils au sein de Worldline}}

  \entry
    {02/2012 à Ajd.}
    {Worldline}
    {Responsable d'application - Unité PST}
    {\emph{Mise en production et exploitation de la plateforme SIG Géoportail v3.Gestion de projet, Lead dév., suivi client, maintien en conditions, évolutions applicatives et systèmes.}}
  \entry
    {04/2011 à 02/2012}
    {Worldline}
    {Responsable d'application - Unité PST}
    {\emph{Exploitation de la plateforme SIG Géoportail v2. Support technique, TMA évolutive, maintien en conditions opérationnelles, MOA système/DBA/Réseau/client.}}

  \entry
    {12/2010 à 04/2011}
    {Worldline}
    {Responsable d'application - Unité PST}
    {\emph{Exploitation de plusieurs plateformes ministérielles, accompagnement technique client, réversibilité de plateforme}}
  \entry
    {12/2010 à 04/2011}
    {Worldline}
    {Responsable d'application - Unité PST}
    {\emph{Exploitation de plusieurs plateformes ministérielles, accompagnement technique client, réversibilité de plateforme}}
  
\end{entrylist}

\section{Formation}

\begin{entrylist}
  \entry
    {since 2009}
    {Ph.D. {\normalfont candidate in Computer Science}}
    {DNET/INRIA, LIP/ÉNS de Lyon}
    {\emph{A Quantified Theory of Social Cohesion.}}
  \entry
    {2007–2008}
    {M.Sc. magna cum laude}
    {IXXI, École Normale Supérieure de Lyon}
    {Majoring in Computer Science\\
    Specialization in Complex Systems}
  \entry
    {2006–2007}
    {B.Sc. magna cum laude}
    {École Normale Supérieure de Lyon}
    {Majoring in Computer Science}
  \entry
    {2003–2006}
    {Classes Préparatoires aux Grandes Écoles}
    {Lycée Fénelon, Lycée Louis le Grand, Paris}
    {Preparation for national competitive entrance exams to leading French ``grandes écoles'', specializing in mathematics and physics.}
  \entry
    {2003}
    {French Baccalauréat S. with honors}
    {Lycée Louis le Grand, Paris}
    {Specialization in mathematics and physics}
\end{entrylist}

\section{interests}

complex networks, social networks, community detection, community structure,
overlapping communities, information diffusion, viral marketing, social
inference, recommendation, data mining

%%% This piece of code has been commented by Karol Kozioł due to biblatex errors. 
% 
%\printbibsection{article}{article in peer-reviewed journal}
%\begin{refsection}
%  \nocite{*}
%  \printbibliography[sorting=chronological, type=inproceedings, title={international peer-reviewed conferences/proceedings}, notkeyword={france}, heading=subbibliography]
%\end{refsection}
%\begin{refsection}
%  \nocite{*}
%  \printbibliography[sorting=chronological, type=inproceedings, title={local peer-reviewed conferences/proceedings}, keyword={france}, heading=subbibliography]
%\end{refsection}
%\printbibsection{misc}{other publications}
%\printbibsection{report}{research reports}

\end{document}
